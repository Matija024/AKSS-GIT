
Almost every noteworthy application has embraced cloud computing. The range of applications that benefit from cloud services spans from small smartphone apps to sophisticated large language models. The clear reason for its popularity is the ability to store, manage, and process data more efficiently than ever before. Cloud services offer numerous advantages, including scalability, flexibility, and cost-effectiveness. However, they also bring significant security challenges. With many critical applications and sensitive data in the cloud, the importance of secure environments has increased. Cloud security involves a diverse range of practices and technologies aimed at protecting data, applications, and services distributed across cloud environments. Basic protection comes from using firewalls and encryption, but cyber threats are getting so advanced and constantly changing that these measures can not always keep up. This calls for advanced security solutions that can ensure the integrity of data even in potentially compromised environments.\\

In order to protect critical sections of programs, Trusted Execution Environments (TEE) were developed. These TEEs have safe areas of memory where essential sections of programs can be executed in isolation from the rest of the system i.e., enclaves. Software Guard Extension (SGX), invented by Intel, falls into the category of a TEE and has Enclaves implemented~\cite{SGX}. These enclaves protect sensitive data and code from being accessed or modified by unauthorized parties, even if the operating system is compromised. SGX provides a robust mechanism for ensuring the confidentiality and integrity of data processed in the cloud. Despite its strengths, Intel SGX still has its vulnerabilities. The enclaves themselves are isolated from adversaries, however, the reliability of the input can not be ensured without additional safety measures. A malicious OS could feed the enclave stale input data, to restore a passed state. Those attacks are labeled as rollback attacks, where data is rolled to an older version~\cite{esccc}. \\
Forking attacks, on the other hand, aim to create multiple instances of an enclave (Clone), leading to unauthorized data access and manipulation. This is accomplished by running the enclave and clones simultaneously and exploiting the fact, that all those instances will return correct but stale states. E.g., counters for password attempts can be reset to gain unlimited tries, despite the limit for tries being set to a finite number. Understanding these attacks and developing effective mitigation strategies is essential for ensuring the security and trustworthiness of cloud-based applications relying on SGX.~\cite{nfw}.\\
In this paper, we will analyze two methods that provide countermeasures against forking attacks. The authors of the first system, Narrator-Pro, relied on a blockchain initialization and developed additional processes for enclaves. The second paper, CloneBusters, on the other hand, created cache-based channels in which enclaves have to communicate, in order to detect clones.

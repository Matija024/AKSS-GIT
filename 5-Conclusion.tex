The reliance on TEE and enclaves provides confidentiality on data and secures executed programs within those enclaves. However, adversaries and malicious OS have found ways to mitigate this reliance by performing attacks on these sealed memory sections. Successful forking attacks on SGX-enabled systems have resulted in acknowledging the threat those systems deliver. Therefore the need for extensions on SGX increased and developers implemented systems that mitigate and prevent forking attacks from happening. 
The system, Narrator-Pro, decided to include a blockchain to initialize enclaves to prevent the launching of clones in the same group. Additional prevention is provided by protocols of state updates, that require SEs and AEs to confirm the freshness of their current state before evolving further. Finally, to ensure that no clone can connect with other groups the restart protocol was introduced.
This implementation delivers comprehensive rules for initialization, updates, and restarts, of enclaves. \\
CloneBusters on the other hand does not prevent clones from launching. It rather relies on defined cache sets in which every instance with the same binary (enclaves and their clones) has to fetch its data. This forces clones to replace data in the cache with their own, thereby enabling CloneBusters to measure cache hits and misses. This recorded data is then sent to a classification algorithm to detect if a clone is running. The cache-based communication is a comparatively simple implementation against forking attacks.\\
Those approaches contribute to the safety of TEEs, enable further development, and encourage the invention of other systems. 